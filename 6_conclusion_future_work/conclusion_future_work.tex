\section{Conclusion and Future Work}
\label{sec:conclusion-and-future-work}

\section{Conclusion}

In this study, a new approach is proposed and tested for generating recommendations using cross-organizational process mining for process performance improvement. Cross-organizational process mining is applied with the idea of unsupervised learning where predictor variables related to performances of organizations are used in an environment where processes are executed on several organizations. Results show that it is possible to use cross-organizational process mining and mismatch patterns for performance improvement recommendations. Process mining is a large-spectrum field where different set of activities and approaches are gathered together to discover, monitor and improve processes. In this  study, a four-stage solution is presented and their performances are explained. In addition, proposed methodology is developed as extensible and configurable set of plugins in ProM framework \cite{verbeek2010prom} and published as open-source. This makes the methodology open to include new process mining methods, mismatch patterns and clustering approaches as well as testing with different datasets.

\section{Future Work}

For the approach proposed in this thesis study, the following issues can be listed as pointers to future work:
\begin{itemize}
	\item In the process mining stage, instead of \textit{Inductive Miner}, different techniques can be used which can mine complex process models with higher appropriateness levels while keeping the current high fitness values.
	\item In the performance indicator analysis stage, new indicators can be defined based on the business environment, event log attributes and user needs. For instance, personnel and resource allocation indicators can be included as well as cost dimension.
	\item For mismatch pattern analysis, new and business oriented mismatch patterns can be included in the analysis. In addition analyzers can fail when there are loops in the process models in current implementations, therefore more robust implementations for process models with loops can be developed in the future.
	\item For the generated recommendations, their quality for business environment is not assessed within the scope of this thesis. However, when any feedback from a domain expert or BPM people is provided, the learning approach can be converted to semi-supervised learning from unsupervised learning.
\end{itemize}