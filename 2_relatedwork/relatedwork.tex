\section{Related Work}
\label{sec:relatedwork}

In this section, studies related to the work presented in this thesis are summarized. Firstly, studies in the process mining area are explained and then studies from cross-organizational process mining, which is the main topic of this research, is introduced. Following these, studies related to similarity in process mining are presented.

Within the process mining framework, there are various different process mining algorithms proposed which have the same aim of discovering underlying processes. Considering the underlying approaches undertaken, algorithms can be grouped as $\alpha$-algorithms \cite{van2004workflow,de2004process}, inductive approaches \cite{herbst1998integrating,herbst2000dealing}, hierarchical clustering \cite{greco2005mining}, genetic approaches \cite{van2005genetic,esgin2010hybrid}, and heuristic approaches \cite{esgin2009hybrid}. Considering the scope of this study; process discovery operations are undertaken with inductive methods which is a robust, repeatable and mature set of approaches.

Cross-organizational mining is based on cross-correlation of workflows and the realized activities in different organization to compare processes and their performances of different organizational units in an objective approach. In the study of Bujis et al. \cite{buijs2012towards}, process models and behaviors of organizations are cross-compared with the idea of supporting and representing. In the studies of van der Aalst \cite{van2011business,van2011intra}, configurable process models for the organizations sharing the same infrastructure and doing the similar work is proposed with the ideas of \textit{exploiting commonality} and \textit{collaboration}. Notion of this study is based on \textit{exploiting commonality} where organizations can learn from each other.

Similarity in process mining have various approaches which focus on metrics \cite{dijkman2011similarity}, analytical comparison \cite{buijs2014comparing}, delta analysis \cite{esgin2011delta,esgin2013sequence} and mismatch patterns \cite{dijkman2007mismatch}. In this research, combination of metric and mismatch pattern approaches are used to identify variations between process models of different organizations.