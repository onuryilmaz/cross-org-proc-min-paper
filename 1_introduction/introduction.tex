\section{Introduction}
\label{sec:intro}

Process mining is a relatively young and developing research area with the roots in computational intelligence, data mining; and process modeling and analysis \cite{van2012process}. Main idea in this research area is to discover, monitor and improve processes by extracting information from event logs. With this idea, process mining creates a bridge between data mining and business process modeling and analysis. Interest in this research area has two origins; firstly, events are recorded and easily available in the modern information systems. Secondly, highly competitive and rapidly changing business life requires improvement and support for business processes \cite{van2012process}. Traditional process mining approaches work on a single organization; however, with the increase of cloud computing and shared infrastructures, event logs of multiple organizations are currently available for analysis. In principle, there are two main environments where cross-organizational process mining stands out. Firstly, when organizations work together to execute the same process, it is insufficient to analyze only logs of one organization and gathered information from all stakeholders should be merged prior to analysis. Secondly, organizations essentially execute the same processes with different needs and configurations on a common infrastructure, where cross-organizational process mining can help the organizations to learn from each other's experience, knowledge and processes.
