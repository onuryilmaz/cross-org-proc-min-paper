\section{Introduction}
\label{sec:introduction}

Process mining is a relatively young and developing research area with the roots in computational intelligence, data mining; and process modeling and analysis \cite{van2012process}. Main idea in this research area is to discover, monitor and improve processes by extracting information from event logs. Traditional process mining approaches work on a single organization; however, with the increase of cloud computing and shared infrastructures, event logs of multiple organizations are currently available for analysis where cross-organizational process mining stands out. In the cross-organizational process mining area, recent studies focus on commonality and collaboration between organizations, especially on how similar the process models and behaviors of organizations under cross comparison are \cite{buijs2012towards} and challenges based on partitioning of tasks and process models between organizations \cite{van2011intra}. This study is based on the environment where processes are executed in several organizations and cross-organizational process mining is applied with the idea of unsupervised learning where predictor variables related to performances of organizations are used. In this environment, underlying assumption of the appraoch is that the correlation between performance values and mismatches hints at a causal relationship.

The approach proposed in this study is a four-stage solution and it starts with mining the process models of organizations; followed by performance indicator analysis and then mismatch pattern analysis. Finally in the suggestion generation stage, learning opportunities are created for each organization. With this approach it is aimed to help business process management users to focus on the potentially important parts of their business maps. Proposed methodology is implemented in ProM framework \cite{verbeek2010prom} as a set of plug-ins corresponding for each stage and packaged under the name of \textit{CrossOrgProcMin} and tested on a synthetic and real-life event logs. Performance of methodology is assessed with a set of defined evaluation metrics for each stage and resulting recommendations are presented to show how this approach helps users to focus on learning opportunities between organizations with a performance improvement potential.

The rest of the paper is organized as follows: In Section~\ref{sec:relatedwork}, related studies in process mining area are presented. In Section~\ref{sec:background}, background information for the relevant topics is explained. In Section~\ref{sec:methodology}, methodology proposed in this study is presented with detail. In Section~\ref{sec:results-and-discussions}, methodology of this study is applied on datasets and results are discussed. In Section~\ref{sec:conclusion-and-future-work}, summary of this study is presented with the final remarks and pointers for future work. 