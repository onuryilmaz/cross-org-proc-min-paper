\documentclass[runningheads,a4paper]{llncs}

\usepackage[american]{babel}

\usepackage{graphicx}

%extended enumerate, such as \begin{compactenum}
\usepackage{paralist}

%put figures inside a text
%\usepackage{picins}
%use
%\piccaptioninside
%\piccaption{...}
%\parpic[r]{\includegraphics ...}
%Text...

%Sorts the citations in the brackets
%\usepackage{cite}

%for easy quotations: \enquote{text}
\usepackage{csquotes}

\usepackage[T1]{fontenc}

%enable margin kerning
\usepackage{microtype}

%better font, similar to the default springer font
\usepackage[%
rm={oldstyle=false,proportional=true},%
sf={oldstyle=false,proportional=true},%
tt={oldstyle=false,proportional=true,variable=true},%
qt=false%
]{cfr-lm}
%
%if more space is needed, exchange cfr-lm by mathptmx
%\usepackage{mathptmx}

%for demonstration purposes only
\usepackage[math]{blindtext}

\usepackage[
%pdfauthor={},
%pdfsubject={},
%pdftitle={},
%pdfkeywords={},
bookmarks=false,
breaklinks=true,
colorlinks=true,
linkcolor=black,
citecolor=black,
urlcolor=black,
%pdfstartpage=19,
pdfpagelayout=SinglePage
]{hyperref}
%enables correct jumping to figures when referencing
\usepackage[all]{hypcap}

\usepackage[capitalise,nameinlink]{cleveref}
%Nice formats for \cref
\crefname{section}{Sect.}{Sect.}
\Crefname{section}{Section}{Sections}
\crefname{figure}{Fig.}{Fig.}
\Crefname{figure}{Figure}{Figures}

\usepackage{xspace}
%\newcommand{\eg}{e.\,g.\xspace}
%\newcommand{\ie}{i.\,e.\xspace}
\newcommand{\eg}{e.\,g.,\ }
\newcommand{\ie}{i.\,e.,\ }

%introduce \powerset - hint by http://matheplanet.com/matheplanet/nuke/html/viewtopic.php?topic=136492&post_id=997377
\DeclareFontFamily{U}{MnSymbolC}{}
\DeclareSymbolFont{MnSyC}{U}{MnSymbolC}{m}{n}
\DeclareFontShape{U}{MnSymbolC}{m}{n}{
    <-6>  MnSymbolC5
   <6-7>  MnSymbolC6
   <7-8>  MnSymbolC7
   <8-9>  MnSymbolC8
   <9-10> MnSymbolC9
  <10-12> MnSymbolC10
  <12->   MnSymbolC12%
}{}
\DeclareMathSymbol{\powerset}{\mathord}{MnSyC}{180}

% correct bad hyphenation here
\hyphenation{op-tical net-works semi-conduc-tor}
 
\begin{document}

%Works on MiKTeX only
%hint by http://goemonx.blogspot.de/2012/01/pdflatex-ligaturen-und-copynpaste.html
%also http://tex.stackexchange.com/questions/4397/make-ligatures-in-linux-libertine-copyable-and-searchable
%This allows a copy'n'paste of the text from the paper
\input glyphtounicode.tex
\pdfgentounicode=1

\title{Recommendation Generation for Performance Improvement by using Cross-Organizational Process Mining}
%If Title is too long, use \titlerunning
%\titlerunning{Short Title}

%Single insitute
\author{Onur Y{\i}lmaz \and P{\i}nar Karag\"{o}z}
%If there are too many authors, use \authorrunning
%\authorrunning{First Author et al.}
\institute{Computer Engineering Department, Middle East Technical University, Turkey}
%\email{yilmaz.onur@metu.edu.tr} 
%\email{karagoz@ceng.metu.edu.tr}
%Multiple insitutes
%Currently disabled
%
\iffalse
%Multiple institutes are typeset as follows:
\author{Firstname Lastname\inst{1} \and Firstname Lastname\inst{2} }
%If there are too many authors, use \authorrunning
%\authorrunning{First Author et al.}

\institute{
Insitute 1\\
\email{...}\and
Insitute 2\\
\email{...}
}
\fi
			
\maketitle

\begin{abstract}
Process mining is a relatively young and developing research area with the main idea of discovering, monitoring and improving processes by extracting information from the event logs. With the increase of cloud computing and shared infrastructures, event logs of multiple organizations are available for analysis where cross-organizational process mining stands with the opportunity for organizations learning from each other. The approach proposed in this study mines process models of organizations and calculates performance indicators; followed by clustering of organizations based on performance indicators and finally spots mismatches between the process models to generate recommendations. This approach is implemented as extensible and configurable plugin set in ProM framework and tested by synthetic and real life logs where successful and suitable results are achieved within evaluation metrics. Generated recommendation results indicate that the use of this approach considerably helps users to focus on the areas of process models with potential performance improvement, which are difficult to spot manually and visually.
\end{abstract}

\keywords{Process Mining, Cross-organizational Process Mining, Recommendation Generation, Process Performance Improvement}


\section{Introduction}
\label{sec:introduction}

Process mining is a relatively young and developing research area with the roots in computational intelligence, data mining; and process modeling and analysis \cite{van2012process}. Main idea in this research area is to discover, monitor and improve processes by extracting information from event logs. Traditional process mining approaches work on a single organization; however, with the increase of cloud computing and shared infrastructures, event logs of multiple organizations are currently available for analysis where cross-organizational process mining stands out. In cross-organizational process mining area, recent studies focus on commonality and collaboration between organizations. Studies focus on how similar the process models and behaviors of organizations under cross comparison \cite{buijs2012towards} and challenges based on partitioning of tasks and process models between organizations \cite{van2011intra}. This study is based on the environment where processes are executed on several organizations and cross-organizational process mining is applied with the idea of unsupervised learning where predictor variables related to performances of organizations are used.

The approach proposed in this thesis is a four-stage solution and it starts with mining the process models of organizations; followed by performance indicator analysis and then mismatch pattern analysis. Finally recommendation generation stage is undertaken to create learning opportunities for each organization. With this approach it is aimed to help business process management users to focus on the potentially important parts of their business maps. Proposed methodology is implemented in ProM framework \cite{verbeek2010prom} as a set of plugins corresponding for each stage and packaged under the name of \textit{CrossOrgProcMin} and tested on a synthetic and real-life event logs. Performance of methodology is assessed with a defined evaluation metrics for each stage and resulting recommendations are presented to show how this approach helps users to focus on learning opportunities between organizations with a performance improvement potential.

The rest of the paper is organized as follows:
\begin{inparaenum}
	\item In Section~\ref{sec:relatedwork}, related studies in process mining area are presented. 
	\item In Section~\ref{sec:background}, background information for the relevant topics is explained.
	\item In Section~\ref{sec:methodology}, methodology proposed in this study is presented with detail.
	\item In Section~\ref{sec:results-and-discussions}, methodology of this study is applied on datasets and results are discussed.
	\item In Section~\ref{sec:conclusion-and-future-work}, summary of this study is presented with the final remarks and pointers for future work. 
\end{inparaenum}
\section{Related Work}
\label{sec:relatedwork}

In this section, studies related to the work presented in this thesis are summarized. Firstly, studies in the process mining area are explained and then studies from cross-organizational process mining, which is the main topic of this research, is introduced. Following these, studies related to similarity in process mining is mentioned. Finally, performance and conformance analysis approaches in process mining area are presented.

 Within the process mining framework, there are various different process mining algorithms proposed which have the same aim of discovering underlying processes. Considering the underlying approaches undertaken, algorithms can be grouped as $\alpha$-algorithms \cite{van2004workflow,de2004process}, inductive approaches \cite{herbst1998integrating, herbst2000dealing}, hierarchical clustering \cite{greco2005mining}, genetic approaches \cite{van2005genetic,esgin2010hybrid}, and heuristic approaches \cite{esgin2009hybrid}. Considering the scope of this study; process discovery operations are undertaken with inductive methods which is a robust, repeatable and mature set of approaches.
 

\bibliographystyle{splncs03}
\bibliography{references/references}


\end{document}
