\section{Background}
\label{sec:background}

In this section, process discovery methods and mismatch patterns are presented within the scope of this work. In process mining field, various process discovery algorithms are proposed to address different challenges in process discovery and using different notations. In this study, since the focus is learning lessons from cross-organizational mining, we used Inductive Process Mining \cite{leemans2013discovering} for process discovery, which is simple, highly applicable and configurable. In the literature, its derivatives which handles infrequent behaviors \cite{leemans2014discoveringinfrequent}; incomplete logs \cite{leemans2014discoveringincomplete}; and model optimization \cite{weidlich2012profiles} are also available. \textit{Inductive Miner Infrequent (IMi)} \cite{leemans2014discoveringinfrequent} extension is used in this study which is capable of filtering the infrequent behavior and results with lower fitness, higher precision and equal generalization.

In cross-organizational process mining environment, there is a need to align processes of different organizations. In order to align these processes and organizations, a mechanism is needed for spotting differences between process models. In the study of Dijkman \cite{dijkman2007mismatch}, a collection of patterns to describe frequent mismatches between the similar process models are presented. Within the scope of this study, the related mismatch patterns are defined in study \cite{dijkman2007mismatch} as follows:
\begin{description}
  \item[Skipped Activity] An activity exists in one process but no equivalent activity is found in the other process.
  \item[Refined Activity] An activity exists in one process but, as an equivalent, a collection of activities are existing in the other process to achieve the same task.
  \item[Activities at Different Moments in Processes] Set of activities are undertaken with different orders in different processes.
  \item[Different Conditions for Occurrence] Set of dependencies are same for two processes; however, occurrence condition is different.
  \item[Different Dependencies] Dependency set of activities differ in different organizations.
  \item[Additional Dependencies] This pattern is a special case of different dependencies where one set of activities includes the other and results with additional dependencies.
\end{description}

As mentioned in the study \cite{dijkman2007mismatch}, their approach does not create a comprehensive list to resolve all mismatches but includes the most common mismatch patterns spotted during case studies. In addition, from their definitions and examples it can be easily seen that these patterns are not orthogonal. Moreover, there are no algorithms provided to spot these mismatches in \cite{dijkman2007mismatch} or consequent studies, and thus implementation of spotting mismatch patterns are performed within the scope of this study.